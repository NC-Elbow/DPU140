\documentclass[16 pt]{amsart}
\usepackage{amscd,amsmath,amsthm,amssymb}
\usepackage{enumerate,varioref}
\usepackage{epsfig}
\usepackage{graphicx}
\usepackage{mathtools}
\newtheorem{thm}{Theorem}
\newtheorem{cor}[thm]{Corollary}
\newtheorem{lem}[thm]{Lemma}
\newtheorem{prop}[thm]{Proposition}
\theoremstyle{definition}
\newtheorem{defn}[thm]{Definition}
\theoremstyle{remark}
\newtheorem{ex}[thm]{Example}
\newtheorem{rem}[thm]{Remark}
\numberwithin{equation}{subsection}
\newcommand{\R}{\mathbb{R}}
\newcommand{\Z}{\mathbb{Z}}
\newcommand{\C}{\mathbb{C}}
\newcommand{\Q}{\mathbb{Q}}
\newcommand{\lh}{\lim_{h\rightarrow 0}}
\begin{document}

\title{Homework 3 Maths 140 Winter 2017 \\ DePaul University\\Dr. Alexander}
\maketitle

\section{Introduction}

In this homework we will begin writing proofs formally. Most of our proofs will come from elementary cases of number theory.


\section{Problems}

Problem 1. Let $n>0$ by a positive integer.  Prove the following:

(a)
\[
3^n - 1 \text{ is even.}
\]

In slightly more sophisticated notation:
\[
2 | (3^n-1)
\]

(b) Either $n^2$ is divisible by 4 or ($\vee$)  $n^2-1$ is divisible by 4.  That is
\[
4 | n^2 \text{ or } 4|(n^2-1)
\]

(c) For all primes $a,b,c$ prove $a^2+b^2 \neq c^2.$

\newpage

Problem 2. We've seen that $\sqrt{2}\notin\Q$.  Let's extend this idea.  Prove the following:\\

(a) $\sqrt{3}\notin \Q$\\

(b) $2^{1/3} \notin \Q$\\

(c) Let $p$ be prime then $\sqrt{p}\notin\Q$.\\

(d) Here's a slightly trickier one.  You'll want to combine the techniques of parts (b) and (c). Let $n>1$ be a positive integer.
\[
p^{1/n} \notin \Q
\]

Hint: Part (d) is really proving something interesting, it's a multitude of facts wrapped up in a single statement, and also says something about the random nature of prime numbers.  Recall:
\[
a|b \iff b = ak \text{ for some } k\in\Z.
\]
That tells us if $p^{1/n} = a/b$ then $p = a^n/b^n$ which tells us $b^n p = a^n$.  So $a^n | b^n$ or $a^n | p$.  Where do we go from here?


\newpage

Problem 3. Prove the following: For all real numbers $a,b$ if $a<b$ then $a<\frac{a+b}{2}<b$

\vspace{1.5in}

Problem 4. Let $a,b,c,d\in\Z$ with $a \neq c$.  Suppose also that $x$ satisfies
\[
\frac{ax+b}{cx+d} = 1.
\]


Is $x\in\Q$?  If so, prove it.  If not give a counterexample.

\newpage

Problem 5. Suppose $a,b,c$ are integers and $x,y,z$ are real so that

\[
\frac{xy}{x+y} = a, \frac{xz}{x+z}=b, \text{ and } \frac{yz}{y+z}=c
\]
Is $x\in\Q$?  If so, prove it.  If not, give a counterexample.


\newpage

\section{Notes}

Problem one gives us the first few techniques in prime testing.  Part (a) is part of a much larger set of divisibilities 
\[
(x-y) | (x^n-y^n)
\]
which we'll explore in more depth in discrete math 2.  Parts (b) and (c) start giving us a little ability to work with parity.  It also sets us up for splitting the integers into more classes than simply even and odd.  Problem 2 moves into some techniques which are useful for prime factoring.  Problem 3 only explores one type of ``average."  In truth there are infinitely many types of averages: Letting $\alpha\in\R-\{0\}$ we have the average at $\alpha$ defined by
\[
Avg_{\alpha}(a,b) = (a^{\alpha}+b^{\alpha})^{1/\alpha}
\]

For example the harmonic average is at $\alpha=-1$
\[
\left(\frac{1}{a}+\frac{1}{b}\right)^{-1}
\]

We also have the geometric average
\[
\sqrt{ab}
\]

The averages themselves also satisfy an inequality:
\[
\alpha<\beta \implies Avg_{\alpha} < Avg_{\beta}
\]


The fraction in problem four is called a M\"{o}bius transformation.  These are used in hyperbolic geometry to move around the hyperbolic plane.  

\end{document}