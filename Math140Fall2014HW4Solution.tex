\documentclass[10 pt]{amsart}
\usepackage{amscd,amsmath,amsthm,amssymb}
\usepackage{enumerate,varioref}
\usepackage{epsfig}
\usepackage{graphicx}
\usepackage{mathtools}
\newtheorem{thm}{Theorem}
\newtheorem{cor}[thm]{Corollary}
\newtheorem{lem}[thm]{Lemma}
\newtheorem{prop}[thm]{Proposition}
\theoremstyle{definition}
\newtheorem{defn}[thm]{Definition}
\theoremstyle{remark}
\newtheorem{ex}[thm]{Example}
\newtheorem{rem}[thm]{Remark}
\numberwithin{equation}{subsection}
\newcommand{\R}{\mathbb{R}}
\newcommand{\Z}{\mathbb{Z}}
\newcommand{\C}{\mathbb{C}}
\newcommand{\Q}{\mathbb{Q}}
\newcommand{\lh}{\lim_{h\rightarrow 0}}
\begin{document}

\title{Homework 4 Maths 140 Autumn 2014}
\maketitle

Section 4.6.12: Prove the following statement by contradiction:\\

``If $a$ and $b$ are rational numbers with $b\neq 0$ and $r$ is irrational, then $a+br$ is irrational."\\

Solution:\\



Section 4.6.16: Prove the following statement by contradiction:\\

``For all odd integers $a,b,$ and $c$, if $z$ is a solution to \[
ax^2+bx+c=0
\]

then $z$ is irrational."\\

Solution:\\


Section 4.6.22: Consider the statement `` For all real numbers $r$, if $r^2$ is irrational then $r$ is irrational."\\

(a) Write what you would suppose and what you would need to prove to show this statement by contradiction.\\

Solution:\\

(b) Write what you would suppose and what you would need to show to prove this statement by contraposition.\\

Solution:\\

Section 10.1.33: Recall $K_6$ denotes a complete graph on 6 vertices.\\
(a) Draw $K_6$.\\

Solution:\\

(b) Show that for all integers $n\geq 1$, the number of edges 
of $K_n$ is $\frac{n(n-1)}{2}$.\\

Solution:\\




Section 10.1.36: Recall that $K_{m,n}$ denotes a complete bipartite graph on $(m,n)$ vertices.\\

(a) Draw $K_{4,2}$\\
Solution:\\

(b) Draw $K_{1,3}$\\
Solution:\\

(c) Draw $K_{3,4}$\\
Solution:\\

(d) How many vertices of $K_{m,n}$ have degree $m$? degree $n$?\\
Solution:\\

(e) What is the total degree of $K_{m,n}$?\\
Solution:\\

(f) Find a formula in terms of $m$ and $n$ for the number of edges of $K_{m,n}$.\\
Solution:\\


\end{document}