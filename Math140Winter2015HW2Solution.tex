\documentclass[16 pt]{amsart}
\usepackage{amscd,amsmath,amsthm,amssymb}
\usepackage{enumerate,varioref}
\usepackage{epsfig}
\usepackage{graphicx}
\usepackage{mathtools}
\usepackage{svg}
\newtheorem{thm}{Theorem}
\newtheorem{cor}[thm]{Corollary}
\newtheorem{lem}[thm]{Lemma}
\newtheorem{prop}[thm]{Proposition}
\theoremstyle{definition}
\newtheorem{defn}[thm]{Definition}
\theoremstyle{remark}
\newtheorem{ex}[thm]{Example}
\newtheorem{rem}[thm]{Remark}
\numberwithin{equation}{subsection}
\newcommand{\R}{\mathbb{R}}
\newcommand{\Z}{\mathbb{Z}}
\newcommand{\C}{\mathbb{C}}
\newcommand{\Q}{\mathbb{Q}}
\newcommand{\lh}{\lim_{h\rightarrow 0}}
\begin{document}

\title{Homework 2 Maths 140 Winter 2015}
\maketitle 

3.1.12: Prove or disprove
\[
 \forall x,y \in \mathbb{R} \sqrt{x+y} = \sqrt{x} + \sqrt{y}
\]

\vspace{1in}

Solution: This is false.  Consider the counterexample $x=2, y=7.$  In fact almost any two numbers will provide a counterexample, provided one of them is not zero. Consider
\[
(\sqrt{x}+\sqrt{y})^2 - \sqrt{x+y}^2= (x + 2xy + y) - (x+y) = 2xy.
\]
That is to say that their difference is always $2xy$ and that difference is only when $xy=0$ or $x=0$ or $y=0$.  Pick any other two numbers and it will provide a counterexample.


\newpage

3.1.18 Let $D$ be the set of all students at your school, and let 

$M(s)$ be ``$s$ is a math major",\\

$C(s)$ be ``$s$ is a computer science student"\\

$E(s)$ be ``$s$ is an engineering student"\\

Express each of the following statements using quantifiers, variables, and predicates\\

a. There is an engineering student who is a math major.\\

Solution: There exists a student $s$ so that $E(s)$ and $M(s)$.  
\[
\exists s \in D \text{ s.t. } E(s)\wedge M(s)
\]

b. Every computer science student is an engineering student.\\

Solution: For every student $s$.  If $C(s)$ then $E(s)$.
\[
\forall s\in D, C(s)\rightarrow E(s)
\]


c. No computer science students are engineering students.\\

Solution: There does not exists a student $s$ so that $C(s)$ and $E(s)$.

\[
\forall s\in D, C(s) \rightarrow \sim E(s) \equiv \sim(\exists s \in D \text{ so that } C(s)\wedge E(s))
\]

Remember: $\sim(p\rightarrow q) \equiv p\wedge \sim q$.\\



d. Some computer science students are also math majors\\

Solution: There exists at least one student $s$ so that $C(s)$ and $M(s)$.
\[
\exists s\in D \text{ so that } C(s)\wedge M(s)
\]


e. Some computer science students are engineering students and some are not.\\

Solution: There exists a student $s$ so that $C(s)$ and $E(s)$.  There is also some other student $t$ so that $C(t)$, but not $E(t)$.
\[
\exists s,t \in D \text{ so that } (C(s)\wedge E(s))\wedge(C(t)\wedge \sim E(t))
\]

\newpage

3.2.8: Consider the statement, ``there are no simple solutions to life's problems." Write an informal negation for the statement, and then write the statement formally using quantifiers and variables

\vspace{1in}

Solution: I prefer to write the statement itself formally first, then negate, then give the informal version.\\

The formal statement is
\[
\sim (\exists s\in S \text{ so that } P(s))
\]
where $S$ is the set of solutions and $P(s)$ is ``$s$ is a simple solution to life's problems."

To negate this we simply add another $\sim$ which leaves us with
\[
(\exists s\in S \text{ so that } P(s))
\]

That is to say ``There is at least one solution to life's problems which is simple.

\newpage

3.2.34: Write the contrapositive for each of the following statements

a. If $n$ is prime, then $n$ is not divisible by any prime number between 1 and $\sqrt{n}$.\\

Solution:\\

If $\sim$($n$ is not divisible by any prime number between 1 and $\sqrt{n}$)\\ 
then $\sim$($n$ is prime).  In other words:

If $n$ is divisible by some number between one and $sqrt{n}$ then $n$ is not prime.\\





b. If $A$ and $B$ do not have any elements in common, then they are disjoint.\\


Solution: \\

If $\sim$($A$ and $B$ are disjoint)\\
 then $\sim$($A$ and $B$ do not have any elements in common).  In other words:\\

If $A$ and $B$ are not disjoint then they share at least one common element.

\newpage


3.4.19:

Rewrite the statement "No good cars are cheap" in the form
"$\forall x$, if $P(x)$ then $\sim Q(x)$." Indicate whether each of the following arguments is valid or invalid, and justify our answers.


a. No good car is cheap

A Rimbaud is a good car

$\therefore$ a Rimbaud is not cheap\\

Let $P(x)$ be ``$x$ is a good car." Let $Q(x)$ be ``$x$ is cheap." Let $R$ be a Rimbaud.\\

Solution: The syllogism in (a) reads


\begin{eqnarray*}
\forall x\in C, & P(x) \rightarrow \sim Q(x)\\
 & P(R)\\
\therefore &  \sim Q(R)
\end{eqnarray*}
This is an example of the universal modus ponens and therefore valid.\\


b. No good car is cheap

A Simbaru is a not cheap

$\therefore$ a Simbaru is a good car.\\

Letting $P(x),Q(x)$ as before and $S$ be a Simbaru, we have\\

\begin{eqnarray*}
\forall x\in C, & P(x) \rightarrow \sim Q(x)\\
& \sim Q(S)\\
\therefore & P(S)
\end{eqnarray*}
This is a converse error.\\




c. No good car is cheap

A VX roadster is a cheap

$\therefore$ a VX roadster is not good\\

\begin{eqnarray*}
\forall x\in C, & P(x) \rightarrow \sim Q(x)\\
 & Q(VX)\\
\therefore & \sim P(VX)
\end{eqnarray*}
This is an example of universal modus tollens.\\


d. No good car is cheap

An Omnex  is not a good car

$\therefore$ an Omnex is cheap\\

\begin{eqnarray*}
\forall x\in C, & P(x) \rightarrow \sim Q(x)\\
 & \sim P(O)\\
\therefore & Q(O)
\end{eqnarray*}

This is an inverse error.


\newpage

4.1.53:

Prove or disprove

For every integer $n$, $n^2-n+11$ is a prime number.

\vspace{1in}

Solution: The easy counterexample is  $n=11 \implies 11^2 - 11 + 11 = 121$ which is not prime.






\end{document}